\documentclass[11pt]{article}
\usepackage{fullpage,amsthm,amsfonts,amssymb,epsfig,amsmath,times,amsthm}
\usepackage{algpseudocode}
\usepackage{tikz}
\usepackage[boxruled,vlined,linesnumbered]{algorithm2e}
\usepackage{ulem}

\newtheorem{theorem}{Theorem}
\newtheorem{claim}[theorem]{Claim}

\theoremstyle{definition}
\newtheorem*{solution}{Solution}
\newtheorem*{algo}{Algorithm}
\newtheorem*{proofcorr}{Proof of correctness}
\newtheorem*{analysis}{Big-O Analysis}

\begin{document}
% PUT YOUR INFORMATION IN THESE TWO LINES 
\hfill FirstName  LastName  

\hfill StudentID  acct@ucsc.edu

\begin{center}
{\bf\Large 
CMPS 102 --- Spring 2020 --  Homework 3 %Solutions (draft version)
}
\end{center}

\begin{center}
Five problems, due Friday May 1 (on canvas) and Monday May 11 (11:59 PM) on Gradescope.  \\
\textit{Due class size, it is likely that only a subset of the problems will be graded.} \\
Ver 0405.2
\end{center}

%\newcommand{\set}[1]{\{ #1 \}}
%\newcommand{\qed}{ \large \hfill $\Box$ \\ \medskip }
%\newcommand{\qedq}{ \large \hfill $\Box$?? \\ \medskip }

% \emph{This is the draft version of written homework two.  
% The questions will not change, but I am distributing it early so 
% you can start thinking about the problems and help catch any ambiguities 
% or typos before the final version of the homework is posted.  } 

\medskip

\renewcommand{\P}{\mbox{IH}}

\noindent
Before you begin the assignment, please read the following carefully.
\begin{itemize}
    \item\textbf{ Read the \emph{Homework Guidelines}}.
    \item Every part of each question begins on a new page. Do not change this.
    \item This does not mean that you should write a full page for every question. Your answers should be short and precise. Lengthy and wordy answers will lose points.
    \item Do not change the format of this document. Simply type your answers as directed.
    \item You are \textbf{not} allowed to work in teams.
\end{itemize}
%Complete the following.
\emph{I have read and agree to the collaboration policy.}  -- FirstName LastName, email@ucsc.edu
% replacing "FirstName LastName, email@ucsc.edu" with your information.
\\
Collaborators: %write the name of the collaborators. If you worked by yourself, please write 'none'.
\\
\hrule
\begin{enumerate}

\item (1 pt) Take the pre-quiz on Canvas by 11 PM on Friday May 1st.  This will be short quiz designed to test
 if you have read and understood the other problems.  The quiz should be available by Thursday afternoon.


\item (6 pts):  Chapter 4, Problem 2. Two true or false statements, with justification (3 pts each).
\begin{solution}
Problem 2.
\begin{itemize}
    \item %answer first question here
    
     \item %answer second question here
\end{itemize}
\end{solution}
\newpage
%Complete the following.
\emph{I have read and agree to the collaboration policy.}  -- FirstName LastName, email@ucsc.edu
% replacing "FirstName LastName, email@ucsc.edu" with your information.
\\
Collaborators: %write the name of the collaborators. If you worked by yourself, please write 'none'.
\\
\hrule
\item (3 + 5 + 5 + 2 = 15 pts). 
The Menlo Park Surgical Hospital admitted a patient, Mr. Banks, who was in a car accident and is still in critical condition and needs continuous monitoring over the next 48 hours (i.e.~all real-valued times between 0 and 48).
At any given time only one nurse needs to be on call for the patient though. 
For this we have an availability interval for each of the $n$ nurses, 
which is a time from which they become available, $a_i$, to the time they must leave for other commitments, $b_i$
%(for $i$ from 1 to $n$).
The $a_i$ and $b_i$ are real valued, and you may assume that for each $i$, $0\leq a_i < b_i \leq 48$.
You need to devise an algorithm that determines a set of nurses to use to cover the next 48 hours of Mr. Banks' stay while having the minimum number of nurses disrupt their normal routine to be on call for him (or report that some time cannot be covered)
A nurse leaving at the same time as another arrives is acceptable. 
Following is a picture of what the intervals for the nurses might look like. 
The darker bars correspond to a set 
of 5 of the 10 nurses who can cover the entire duration. (Notice, though, that 4 nurses would have sufficed.)

\begin{center}
	\includegraphics[height=1in]{subcover}
\end{center}

Your efficient {\bf greedy} algorithm should take as input a list of pairs of times $(a_i,b_i)$ for $i=1$ to $n$ and the end time $T$.

\textit{Continued on next page.}
\newpage
\begin{enumerate}
	\item Consider the greedy algorithm that selects nurses by
	repeatedly choosing the nurse who will be there for the longest
	time among the periods not covered by previously selected
	nurses. For example,if times 2 through 8 are covered by already selected nurses a another nurses interval is (5,10),
	then adding that nurse would only provide additional coverage form 8 to 10, or 2 units of time. 
	Give an example showing that this algorithm does \textbf{not}
	always find the smallest set of nurses.
	\begin{solution}
	
	\end{solution}
	\newpage
	\item Present an algorithm that always outputs a smallest subset of nurses that can cover the entire 48 hours 
	or report that no such subset exists.
	\begin{solution}
	
	\end{solution}
	\newpage
	\item Prove that your algorithm is correct (i.e.\ alway finds a smallest possible subset of the nurses the cover the entire stay).
	\begin{solution}
	
	\end{solution}
	\newpage
	\item State its running time with a brief one-to-three sentence justification. 
	\begin{solution}
	
	\end{solution}
	\newpage
\end{enumerate}
%Complete the following.
\emph{I have read and agree to the collaboration policy.}  -- FirstName LastName, email@ucsc.edu
% replacing "FirstName LastName, email@ucsc.edu" with your information.
\\
Collaborators: %write the name of the collaborators. If you worked by yourself, please write 'none'.
\\
\hrule
\item (15 pts) A first grade teacher has a set of $n$ books, say $\{1, 2, \ldots, n\}$, stored on a bookshelf in the classroom.
Each book $k$ has a popularity represented by the probability $p(k)$ of being wanted by a student (since these are probabilities they are non-negative and sum to 1). 
Unfortunately, the students are not sophisticated enough to do binary search, and so do a sequential search 
for the book from left to right.   When $s_1$ is the first book on shelf, $s_2$ the second, and so on, 
and some book $s_i$ is wanted, it will time $i$ for the student to locate it. 
When all the books are on the shelf, the average search time is $\sum_{i=1}^n i \cdot p(s_i)$.  
The problem is to find (and prove correct) a greedy algorithm that that takes the $p(k)$ values and 
determines an order $s_1, s_2, \ldots , s_n$ for storing the books on the shelf that minimizes this average search time.

(Hint: Use the structure of the problem or an exchange argument to prove your greedy algorithm's solution is optimal.)

\textit{Continued on next page.}
\newpage
%Answer beings on a new page.
\begin{algo}


\end{algo}
\newpage
\begin{proofcorr}

\end{proofcorr}
\newpage
\begin{analysis}

\end{analysis}
\newpage

%Complete the following.
\emph{I have read and agree to the collaboration policy.}  -- FirstName LastName, email@ucsc.edu
% replacing "FirstName LastName, email@ucsc.edu" with your information.
\\
Collaborators: %write the name of the collaborators. If you worked by yourself, please write 'none'.
\\
\hrule
\item (15 pts)  Is it a break-in?

The security staff for a high-tech company suspects that a notorious hacker is breaking into their network and stealing their trade secrets.   
The traffic on the network over the last couple of days is represented by a sequence of 
events $E=(e_1, e_2, \ldots, e_n)$
where each $e_i$ is an integer event ID.  
These IDs can come from all sorts of applications, and the same event number can be seen multiple times.
The notorious hacker has a predictable way to break in:  he causes a particular 
sequence of $k<n$ events $H=(h_1, h_2, \ldots, h_k)$ to happen on the network to help him gain access.
These events have to occur in that particular order, and some event numbers may be be repeated in $H$.
Your task is to design and prove correct a greedy algorithm for determining if $H$ is a subsequence of $E$
\uline{and when the hacker could have broken in.}
Recall that $H$ is a subsequence of $E$ if $H$ is embedded in $E$, i.e.~some number (possibly 0) of the events in $E$ can be deleted so that your are left with $H$.  
More formally, the $k$ length sequence $H$ is a subsequence of $E$ if there is a sequence of $k$ indices 
$i_1 < i_2 < \cdots < i_k$ such that each $h_j = e_{i_j}$.

For example, neither (2, 4, 5) nor (4, 2, 2, 5) are subsequences of 
(1, 9, 4, 1, 4, 2, 6, 5, 7), but (9,5) and (4, 4, 2, 5) are.

\uline{
Design a greedy algorithm running in $O(n)$ time that determines if $H$ is a subsequence of $E$ and returns
the first (lowest) $\ell$ such that $H$ is a substring of $E_\ell = e_1, e_2, \ldots, e_\ell$ 
(i.e.~$E_{\ell}$ is the shortest prefix of $E$ that contains $H$ as a substring).
Prove that your algorithm is correct and briefly justify its running time.
}

\uline{
To prove correctness, it may be helpful for your algorithm to compute the sequence of indices where
the events in $E$ match those in $H$.
}

\textit{Continued on next page.}
\newpage
%Answer beings on a new page.
\begin{algo}


\end{algo}
\newpage
\begin{proofcorr}

\end{proofcorr}
\newpage
\begin{analysis}

\end{analysis}
\newpage

\end{enumerate} 
\end{document}
