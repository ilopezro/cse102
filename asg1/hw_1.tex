\documentclass[11pt]{article}
\usepackage{fullpage,amsthm,amsfonts,amssymb,epsfig,amsmath,times,amsthm}
\usepackage{algpseudocode}


\newtheorem{theorem}{Theorem}
\newtheorem{claim}[theorem]{Claim}
\newtheorem*{solution}{Solution}

\begin{document}
% PUT YOUR INFORMATION IN THESE TWO LINES 
\hfill FirstName  LastName  

\hfill StudentID, acct@ucsc.edu

\begin{center}
{\bf\Large 
CMPS 102 --- Spring 2020 --  Homework 1 }
\end{center}

\begin{center}
Updated Ver. 3(04-05)\\
Four problems, 25 points, due Friday April 10 (11:59 PM) on Gradescope. 
\end{center}

%\newcommand{\set}[1]{\{ #1 \}}
%\newcommand{\qed}{ \large \hfill $\Box$ \\ \medskip }
%\newcommand{\qedq}{ \large \hfill $\Box$?? \\ \medskip }

\renewcommand{\P}{\mbox{IH}}
Before you begin the assignment, please read the following carefully.
\begin{itemize}
    \item Read the \emph{Homework Guidelines}.
    \item Every part of each question begins on a new page. Do not change this.
    \item This does not mean that you should write a full page for every question. Your answers should be short and precise. Lengthy and wordy answers will lose points.
    \item Do not change the format of this document. Simply type your answers as directed.
    \item You are \textbf{not} allowed to work in teams.
\end{itemize}
%Complete the following.
\emph{I have read and agree to the collaboration policy.}  -- FirstName LastName, email@ucsc.edu
% replacing "FirstName LastName, email@ucsc.edu" with your information.
\\
Collaborators: %write the name of the collaborators. If you worked by yourself, please write 'none'.
\\
\hrule
\begin{enumerate}
\item (Total: 8 pts) A grad student comes up with the following algorithm to sort an array $A[1..n]$ that works by first sorting the first 2/3rds of the array, then sorting the last 2/3rds of the (resulting) array,
and finally sorting the first 2/3rds of the new array.

\begin{algorithmic}[1]
\Function{G-sort}{$A$, $n$} \Comment {takes as input an array of $n$ numbers, $A[1..n]$}
	\State G-sort-recurse($A$, 1, $n$)
\EndFunction

 \Function {G-sort-recurse}{$A$, $\ell$, $u$} % \Comment {sort subarray $A[\ell..\u]$ } 
\If {$u - \ell \leq 0$} 
	\State return \Comment{1 or fewer elements already sorted}
\ElsIf{$u-\ell = 1$} \Comment{2 elements}
	\If {$A[u] < A[\ell]$ } \Comment{swap values}
		\State  temp $\gets A[u]$
		\State $A[u] \gets A[\ell]$
		\State $A[\ell] \gets$ temp
	\EndIf
\Else \Comment{3 or more elements}
\State size $\gets u - \ell + 1$
\State twothirds $\gets \lceil (2 * \mbox{size})  / 3 \rceil$
\State G-sort-recurse($A, \ell, \ell + \mbox{twothirds} - 1$)
\State G-sort-recurse($A, u - \mbox{twothirds}+1 , u$)
\State G-sort-recurse($A, \ell, \ell + \mbox{twothirds} - 1$)
\EndIf 
\EndFunction
\end{algorithmic}

a (4 pts). First, prove that the algorithm correctly sorts the numbers in the array (in increasing order).  
After showing that it correctly sorts 1 and 2 element intervals, you may make the (incorrect) assumption that the number of elements being passed to \emph{G-sort-recurse} is always a multiple of 3 to simplify the notation (and drop the floors/ceilings). 
\begin{solution}
%type your answer here
\end{solution}
\newpage

b (1 pts). Next, Derive a recurrence for the algorithm's running time (or number of comparisons made).
\begin{solution}
%type your answer here
\end{solution}
\newpage

c (3 pts). Finally, obtain a good asymptotic upper bound (big-$O$) for your recurrence.  
\begin{solution}
%type your answer here
\end{solution}
\newpage
%Complete the following.
“I have read and agree to the collaboration policy.” -- FirstName LastName, email@ucsc.edu
\\
Collaborators: %write the name of the collaborators. If you worked by yourself, please write 'none'.
\\
\hrule


\item (Total: 8 pts) Induction Proof correctness.

Recall that a full binary tree 
contains (A) just a single leaf node, or (B) is
an internal node (the root) connected to two disjoint 
subtrees, which are themselves full binary trees.

First consider the following claim and proof.  First, think about if the theorem is true or not, and if the proof is correct or not
(do not include these preliminary thoughts in your answer).

\begin{claim} 
In any full binary tree, the number of leaf nodes is 
one greater than the number of internal nodes.
\end{claim}

\begin{proof} (??)  By induction on number of internal nodes.

For all $n\geq 0$, let $\P(n)$ be the statement:
``all full binary trees having exactly 
$n$ internal nodes have $n+1$ leaf nodes.''

{\bf Base Case:} Show $\P(0)$. 
Every full binary tree with zero internal nodes is formed by
case (A) of the definition, and thus consists of 
just a single leaf node.  
Therefore, every full binary tree
with 0 internal nodes has exactly 1 leaf node, and $\P(0)$ is true.

{\bf Inductive Step:} Assume $k>0$ and $\P(k)$ holds to show that $\P(k+1)$ also holds. 

Consider an arbitrary full binary tree $T$ with $k$ internal nodes. 
By the inductive hypothesis $T$ has  $k+1$ external nodes.
Create a $k+1$ internal node tree $T'$ by removing a bottom leaf node in $T$ and replacing it with an internal node connected to
two children that are leaves.
$T'$ has one more internal node than $T$, and $2-1=1$ more external node than $T$.
Therefore $T'$ has $k+1$ internal nodes and $(k+1)+1=k+2$ external nodes, proving $P(k+1)$.
\end{proof}

Now consider the following claim.  
\begin{claim} \label{c:height}
For all $n$, all full binary trees with $n$ internal nodes have
height $n-1$.
\end{claim}

We have the following ``proof'' of the claim.

\begin{proof} ??
By induction on $n$. For each $n\geq 1$,
let $\P(n)$ be the statement ``all  full binary trees with $n$ internal nodes have height $n-1$''.

{\bf Base Case:} Show $\P(0)$. 
Every  full binary tree with zero internal nodes is formed by
case (A) of the definition, and thus consists of 
just a single leaf node.  
Therefore, every full binary tree
with 0 internal nodes has only one leaf (which is also the root).
Thus the longest root-to-leaf path has length 0, and $\P(0)$ is true.

{\bf Inductive Step:} Let $n\geq 1$ and show that $\P(n)$ implies $\P(n+1)$.
Let $T$ be an arbitrary full binary tree with $n$ internal nodes.  
Create the binary tree $T'$ having $n+1$ internal nodes  
by removing a bottom leaf node in $T$ and replacing it with an internal node connected to
two children that are leaves.
The longest root-to-leaf path in $T'$ is thus one greater than the longest root-to-leaf path in $T$,
so the height of $T'$ is the height of $T$ plus 1.
Furthermore, by the inductive hypothesis $\P(n)$, the height of $T$ is $n-1$.
Therefore $T'$ has $n+1$ internal nodes and height $n-1+1=n$, showing $\P(n+1)$.
\end{proof}


a (1 pt). Give a counter-example showing that claim~\ref{c:height} is false.
(Recall that the height of a binary tree is the length of the longest root-to-leaf path).
\begin{solution}
%type your answer here
\end{solution}
\newpage

b (3 pts). Identify and clearly describe the flaw in the proof of claim~\ref{c:height}.
\begin{solution}
%type your answer here
\end{solution}
\newpage
c (4 pts). Now go back to the proof of the first claim.  \\
Does it have a flaw or hidden assumption? (1 pt) \\
Either clearly describe the flaw/assumption, or argue that the proof is correct (3 pts).
\begin{solution}
%type your answer here
\end{solution}
\newpage
%Complete the following.
“I have read and agree to the collaboration policy.” -- FirstName LastName, email@ucsc.edu
\\
Collaborators: %write the name of the collaborators. If you worked by yourself, please write 'none'.
\\
\hrule
\item (Total: 6 pts) Asymptotic notation:  Exercise 5 of Chaper 2: Prove or disprove 3 asymptotic implications: If $f(n)$ is in $O(g(n))$ then is it always true that: \\
a (2pts). $\log_2 (f(n))$ is in $O(( \log_2 (g(n))$.  
\begin{solution}
%type your answer here
\end{solution}
\newpage
b (2 pts). $2^{f(n)}$ is in $O( 2^{g(n)} )$. 
\begin{solution}
%type your answer here
\end{solution}
\newpage
c (2 pts). $f(n)^2$ is in $O(g(n)^2)$.
\begin{solution}
%type your answer here
\end{solution}
\newpage
%Complete the following.
“I have read and agree to the collaboration policy.” -- FirstName LastName, email@ucsc.edu
\\
Collaborators: %write the name of the collaborators. If you worked by yourself, please write 'none'.
\\
\hrule
\item (3 pts) Least counterexample: Give a proof by contradiction using the \emph{least counterexample} method that:
\[
\sum_{i=0}^n i = \frac{n (n+1)}{2}  \qquad \text{for all $n \geq 0$}.
\]
\begin{solution}
%type your answer here
\end{solution}
\newpage
\end{enumerate}

\subsection*{Recommended exercises (not to be turned in)}
\begin{enumerate}
\item The solved exercises in Chapters 1, 2, and 5 (Divide and Conquer).
\item Exercises 1 and 2 in Chapter 1 of the text.
\item Exercises 3 and 4 in Chapter 2 of the text.
\end{enumerate} 
\end{document}
